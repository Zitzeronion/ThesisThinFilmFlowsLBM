\section*{Prolog}
\label{sec:prolog}
My time as a student hopefully and finally ends with this work, my thesis.
I started studying physics quite some time ago and honestly doing research for my doctorate feels very different from what university was before.
While there are many things to write here as objective as possible, most of it is still tainted with my style.
Therefore I want to use this prologue as a short but important personal summary.
Working on problems nobody has solved or thought about is hard.
Creativity is an essential skill to perform research.
Doing what everybody does is easy, doing what other people tell you to do is easy.
Finding my very own way to approach problems and designing computational experiments was unbelievable hard work. 
Not only did I do far more things wrong than right, but I also had a naive and idealistic view of science.
But science is done by humans and not by perfectly objective entities\footnote{Any AI, as of today, is far away from being an objective entity.}.
Doing research can be fairly disappointing on the one hand or very rewarding on the other.
For me it was mostly the former and while failures or null results are an integral part of science they rarely make into publications.
A situation which at least from my point of view does not improve the quality of research.
It is well known that most of academia's staff is showing symptoms of burnout and are exposed to constant elevated levels of stress~\cite{gewin2021pandemic, bilge2006examining, henny2014prevalence, watts2011burnout, satinsky2021systematic}.
I am just a PhD student, but even I felt guilty during weekends for enjoying the weekend.
Is this a sustainable way to educate and far more important is this a healthy work environment?
One could say, ``well the first three years are bad, but then it gets better``. 
Truth is that the situation of scientists in academia is not optimal, where chain contracts are only one issue.
The requirement of moving to different countries, acquiring funds and publishing research all at the same time is by all means an unhealthy one.

While this may sound depressing, I think it is an opportunity.
An opportunity to make things better, healthier and the first step is to talk openly about the situation.
That is why I choose to add quotes at the start of every chapter.
These quotes are not famous words from important people, they are parts of lyrics from pop songs.
Songs that highlight that our society is flawed.
Singing about problems such as the requirement to be perfect out of the box or the constant repression of minorities for example.
I think these lyrics are quite fitting to describe the state I found myself in during the time as a student and even more so during the writing of this thesis.