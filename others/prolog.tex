\section*{Prolog}
\label{sec:prolog}
My time as a student hopefully and finally ends with this work, my thesis.
I started studying physics quite some time ago and honestly doing research for my doctorate feels very different from what university was before.
While there are many things to write here as objective as possible, most of it is tainted with my style.
Therefore I want to use this prologue as a short but to me important personal summary.
Working on problems nobody has solved or thought about is hard.
Creativity is an essential skill to perform research.
Often it is easy to do the same over and over again, most of the time following orders is also easy.
Finding my very own way to approach problems and designing computational experiments was unbelievable hard work. 
Not only did I do far more things wrong than right, but I also had a naive and idealistic view of science.
But science is done by humans.
Doing research can be fairly disappointing on the one hand but also very rewarding on the other.
For me it was mostly the former and while failures or null results are an integral part of science they are rarely communicated.
A situation which at least from my point of view does not improve the quality of research and in fact does inevitably lead to recurring mistakes and thus wasted effort.

I am just a PhD student, but even I felt guilty during weekends for not working and instead engaged in leisure activities.
Why did I feel guilty, because projects did not work the way they should have and results did not agree with hypotheses.
That put pressure on me, because whatever we do we want to see progress.
If nothing adds up and you hit a dead end after another it is hard to see progress.

Of course my situation was still good, I for example didn't have to acquire funding for my position.
Truth is that the situation of scientists in academia is not optimal and chain contracts are only one issue.
The requirement of moving to different countries, acquiring funds and publishing research all at the same time is by all means an unhealthy one.
It is also remarkable that plenty of studies showed that the situation is far from being good, yet scientists world wide accept the status quo.

While this may sound depressing, I think it is an opportunity.
An opportunity to make things better, healthier and the first step is to talk openly about the situation.
That is why I choose to add quotes at the start of every chapter.
These quotes are not famous words from important people, they are parts of lyrics from pop songs.
Songs that highlight that our society is flawed.
Singing about problems such as the requirement to be perfect out of the box or the constant repression of minorities for example.
I think these lyrics are quite fitting to describe the state I found myself in during the time as a student and even more so during the writing of this thesis.
