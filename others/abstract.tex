\begin{abstract}
\section*{Abstract}
Fluid dynamics is a part of our everyday life.
The governing equations describe not only large weather systems on planetary scales ($10^6$m) but also reach down to wetting problems on the nanometric scale ($10^{-9}$m). 
That a single theory can describe both the movement of atmospheric layers as well as the displacement of the contact line of a droplet with a diameter smaller than a micrometer ($10^{-6}$m) is highly non trivial and related to the scale invariant nature of flow problems.

The lubrication approximation was first thought of by Osborn Reynolds in the $19^{\text{th}}$ century. 
He understood that a lubrication layer is essentially a thin layer of fluid between mechanical components.
This thin layer offered some novelties which made it possible to simplify the Navier-Stokes equation.
From that time onwards the functional formulation of the lubrication approximation, the so-called thin film equation, has shed new light on problems such as wetting, coating and printing to name but a few. 

Today the thin film equation is used to understand several phenomena in nature and on the other hand to optimize various industrial applications.
Insects for example need to have a water-repellent exoskeleton. 
Due to their size they would not be able to overcome the surface tension of water and therefore could become trapped inside a drop.
In agriculture the distribution of pesticides relies on the adhesion between the pesticide's transport fluid and the plant's surface. 
Having a too low affinity would just wash off the pesticide, however by reducing the surface tension via so-called surfactants it is more likely that the pesticide sticks to the plant.
Another application relates to the production of computer chips.
The process itself is fairly complex, however at some point the yield on the wafer heavily relies on the homogeneous distribution of photoresist, and therefore on wetting.

Over the past six decades numerical methods have become more and more essential to study fluid dynamics. 
Culminating in a topic of research of its own called computational fluid dynamics (CFD). 
Not only is it used in an academic environment but also in areas that do not come to mind at first glance, e.g. computer games and animated movies.
One numerical approach among the vast zoo of CFD methods is the lattice Boltzmann method (LBM).
Instead of a direct simulation of the fluid's density and velocity the LBM uses a statistical approach based on the Boltzmann equation. 
We make use of this approach and derive a model that is capable of solving the thin film equation.
The starting point is the shallow water lattice Boltzmann method. 
Interestingly enough with well defined approximations the shallow water dynamics can be recast to the thin film equation.
This approach is tested and in agreement with known numerical and experimental results such as instabilities, droplet dynamics and dewetting.

Dewetting, i.e. the process when a liquid retracts from a surface, is extensively studied in this thesis. 
When it comes to dewetting one mostly ignored effect of experiments is the presence of thermal fluctuations.
Theoretically this poses a different and more complex problem as the resulting equation is no longer deterministic.
The resulting stochastic thin film (STF) equation can however be treated with the here developed LBM.
Including fluctuations do in fact accelerate dewetting, which is in agreement with previous studies, but can also lead to novel couplings when the substrates' wettability is heterogeneous.\\
Dewetting can also be addressed using a time and spatially dependent wettability.
The priority program SPP 2171 of the German Research Foundation (DFG) aims to deepen our understanding of liquid-substrate interactions.
Our contribution to this program is the study of so-called switchable substrates.
With simple arguments it is shown that a morphological transition during dewetting can be induced. 
Instead of several droplets fewer rivulets are present based on the dynamics of the local wettability.

\pagebreak
\section*{Kurzfassung}
Hydrodynamik, als solches die Bewegung von Flüssigkeiten, ist ein wichtiger Teil unseres Alltags.
Die definierenden Gleichungen beschreiben nicht nur die Dynamik von Wettersystemen planetarer Ordnung ($10^6$m), sondern auch die Benetzung fein strukturierter Oberflächen ($10^{-9}$m).
Es ist erstaunlich, dass die Theorie der Hydrodynamik sowohl die Bewegung von atmosphärischen Schichten als auch die Verschiebung der Kontaktline eines Tropfens mit einer Größe kleiner als einem Mikrometer ($10^{-6}$m) beschreiben kann.
Diese Eigenschaft der Skaleninvarianz macht die Hydrodynamik zu einem äußerst interessanten Forschungsgebiet.

Osborn Reynolds hatte sich bereits im 19. Jahrhundert mit der Frage der Schmiermittelreibung beschäftigt.
Dabei stellte er fest, dass ein Schmiermittel einen dünnen Film zwischen den Mechanischen Komponenten bildet.
Diese Tatsache, dass der Film nur eine geringe Höhe zwischen den Komponten einnehmen konnte macht es möglich die Navier-Stokes Gleichung zu vereinfachen.
Diese Lubrikationsapproximation fand ihre Anwendung in der Dünnfilmgleichung, die seit mehr als 100 Jahren neues Licht auf diese vermeintlich einfachen Probleme wirft.

Heutzutage wird die Dünnfilmgleichung genutzt, um Phänomene in der Natur zu studieren und um industrielle Prozesse zu optimieren.
Insekten würden zum Beispiel in einem Wassertropfen ertrinken wenn, Sie von einem Tropfen benetzt würden. 
Als Antwort auf dieses Problem ist ein Insektenpanzer nur schwer benetzbar oder einfacher gesagt ``wasserabweisend''.
In der Landwirtschaft wiederum sollten Pestizide an den Pflanzen haften und nicht einfach im Boden versickern.
Ob die Pestizide an der Pflanze haften oder nicht hängt von den Benetzungseigenschaften der Trägerflüssigkeit des Pestizids ab. 
Um eine bessere Benetzung zu garantieren, wird mit Oberflächenspannungsreduzierenden Komponenten, sogenannten Tensiden, gearbeitet.
Ein anderer industrieller Sektor, der auch noch in Zukunft wichtiger werden wird, beschäftigt sich mit der Herstellung von Computerchips.
Der Prozess vom Silizium Einkristall zum logischen Schaltkreis ist hochkomplex, aber an einem Punkt der Produktion wird per Lithographie eine Struktur aufgedruckt. 
Dieser Schritt ist unter anderem dadurch begrenzt wie gleichmäßig sich einen bestimmte chemische Beschichtung, der sogenannte Photoresist, am Wafer verteilt hat. 
Ein Problem das zwar auf den ersten Blick nicht viel mit dem Verteilen von Pestiziden gemein hat, aber dessen Problemstellung auch im Gebiet der Benetzungsprobleme angesiedelt ist.

In den letzten sechzig Jahren sind numerische Methoden zur Untersuchung von hydrodynamischen Problem immer wichtiger geworden.
Kulminiert ist dieser Trend in einem eigenen Forschungsfeld der numerischen Strömungsmechanik, oder colorful fluid dynamic (CFD).
Aber auch außerhalb der Forschung findet die CFD ein breites Anwendungsspektrum.
Zum einen werden CFD Simulationen genutzt um Flugzeugflügel oder Motoren von Kraftfahrzeugen zu optimieren, zum anderen findet CFD auch Anwendung in Filmen und Computerspielen.
Eine von vielen Methoden der numerischen Strömungsmechanik ist die Gitter Boltzmann Methode (LBM), die sich der Boltzmann Gleichung der Statistischen Physik bedient.
Um die Dynamik dünner Flüssigkeitsfilme mit der LBM zu simulieren bedarf es einiger Annahmen und Erweiterungen.
Ausgangspunkt für unser Model ist die LBM für seichte Gewässer (shallow waters). 
Mit ein paar Annahmen und etwas Algebra lässt sich das Gleichungssystem des seichten Gewässers in die Dünnfilmgleichung umformen.
Das so entstandene Model wird mit bekannten Problem der Dünnfilmgleichung getestet und validiert.
Diese ``benchmark''-Probleme reichen von Instabilitäten über Tropfendynamik bis hin zur Entnetzung von Oberflächen.

Entnetzung wird hier verstanden als das aufbrechen eines Flüssigkeitsfilms.
Anstelle einer homogenen Flüssigkeitsschicht entstehen beim Entnetzen einzelne Tropfen. 
Im Laufe dieser Arbeit wird Entnetzung ein immer wiederkehrendes Thema sein.
Üblicherweise wird in Entnetzungsexperimenten die Rolle der thermischen Fluktuationen nicht berücksichtigt.
Der hier definierte Ansatz macht es möglich, mit simplen Erweiterungen deren Effekt zu simulieren und zu klassifizieren.
Die somit gewonnen Erkenntnisse können vorherige Studien reproduzieren und teilweise ergänzen.

Entnetzung von Flüssigkeiten kann viele Gründe haben, zum Beispiel durch räumliche und zeitliche Änderungen der Benetzbarkeit der Oberfläche.
Dieses Problem ist in keinster Weise trivial was sich auch darin äußert, dass die Deutsche Forschungsgemeinschaft (DFG) ein Schwerpunktsprogramm fördert welches sich unter anderem mit dieser Fragestellung widmet. Konkret beschäftigt sicher der SPP 2171 mit dem dynamischen Benetzen von flexiblen, adaptiven and schaltbaren Substraten.
Deswegen findet sich in diese Arbeit auch der Themenkomplex der schaltbaren Substrate, also Oberflächen die ihre Benetzungseigenschaft durch z.B. Beleuchtung ändern.
Mit einem einfachen Modell wird gezeigt, dass es möglich ist die Entnetzungsmorphologie des dünnen Films zu ändern.
Anstatt von Tropfen, wie es zu vermuten wäre auf einem gemusterten Substrate, werden Bandstrukturen gebildet.
Dieser bisher unerforsche Übergange zu Bandstrukturen hängt nicht nur von den schaltbaren Eigenschaften des Substrates ab, sondern auch von z.B der Oberflächenspannung der Flüssigkeit.

\end{abstract}
