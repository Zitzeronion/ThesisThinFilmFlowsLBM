\begin{abstract}
\textbf{Abstract}
Thin liquid film flows are encountered in our every days life in many occasions.
Starting from large weather systems on planetary scales reaching down to wetting problems on the nano scale. 
This so called scale invariance is what makes the thin film equation or in other words, the lubrication approximation a valuable tool to study flows.

The lubrication approximation was first thought of by Osborn Reynolds as he understood that in the case of a small third dimension the Navier-Stokes equations can be simplified.
Since then the thin film equation has shed new light on fluid dynamics. 

Today the thin film equation is used to understand several process in nature and industry. 
Especially in the low Reynolds number regime (Re$<1$) a vast amount of problems can be addressed with this approach. 
Just to name a few, the classical coating problem can be addressed with the thin film equation.
On the other hand many droplet related problems are in fact solveable with the thin film equation.

Over the past two decades numerical techniques have been developed to study this kind of systems.
Within this work a novel approach based on the lattice Boltzmann method (LBM) will be introduced.
This numerical tool is tested against known thin film results, such as instabilities, substrate dewetting and droplet flows.
Further new insights on the stochastic thin film equation will be shown and discussed.

\end{abstract}
