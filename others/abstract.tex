\begin{abstract}
\section*{Abstract}
Fluid dynamics is a part of our everyday life.
The governing equations describe not only large weather systems on planetary scales ($10^6$m) but also reach down to wetting problems on the nanometric scale ($10^{-9}$m). 
That a single theory can describe both the movement of atmospheric layers as well as the displacement of the contact line of a droplet with a diameter smaller than a micrometer ($10^{-6}$m) is highly non trivial and related to the scale invariant nature of flow problems.

The lubrication approximation was first thought of by Osborn Reynolds in the $19^{\text{th}}$ century. 
He understood that a lubrication layer is essentially a thin layer of fluid between mechanical components.
This thin layer offered some novelties which made it possible to simplify the Navier-Stokes equation.
From that time onwards the functional formulation of the lubrication approximation, the so-called thin film equation, has shed new light on problems such as wetting, coating and printing to name but a few. 

Today the thin film equation is used to understand several phenomena in nature and on the other hand to optimize various industrial applications.
Insects for example need to have a water-repellent exoskeleton. 
Due to their size they would not be able to overcome the surface tension of water and therefore could become trapped inside a drop.
In agriculture the distribution of pesticides relies on the adhesion between the pesticide's transport fluid and the plant's surface. 
Having a too low affinity would just wash off the pesticide, however by reducing the surface tension via so-called surfactants it is more likely that the pesticide sticks to the plant.
Another application relates to the production of computer chips.
The process itself is fairly complex, however at some point the yield on the wafer heavily relies on the homogeneous distribution of photoresist, and therefore on wetting.

Over the past six decades numerical methods have become more and more essential to study fluid dynamics. 
Culminating in a topic of research of its own called computational fluid dynamics (CFD). 
Not only is it used in an academic environment but also in for entertainment purposes, e.g. computer games and animated movies.
One numerical approach among the vast zoo of CFD methods is the lattice Boltzmann method (LBM).
Instead of a direct simulation of the fluid's density and velocity the LBM uses a statistical approach based on the Boltzmann equation. 
We make use of this approach and derive a model that is capable of solving the thin film equation.
The starting point is the shallow water lattice Boltzmann method. 
Interestingly enough with well defined approximations the shallow water dynamics can be recast to the thin film equation.
This approach is tested and in agreement with known numerical and experimental results such as instabilities, droplet dynamics and dewetting.

Dewetting, i.e. the process when a liquid retracts from a surface, is extensively studied in this thesis. 
When it comes to dewetting one mostly ignored effect of experiments is the presence of thermal fluctuations.
Theoretically this poses a different and more complex problem as the resulting equation is no longer deterministic.
The resulting stochastic thin film (STF) equation can however be treated with the here developed LBM.
Including fluctuations does in fact accelerate dewetting, which is in agreement with previous studies, but can also lead to novel couplings when the substrates' wettability is heterogeneous.\\
Dewetting can also be addressed using a time and spatially dependent wettability.
The priority program SPP 2171 of the German Research Foundation (DFG) aims to deepen our understanding of liquid-substrate interactions.
Our contribution to this program is the study of so-called switchable substrates.
With simple arguments it is shown that a morphological transition during dewetting can be induced. 
Instead of several droplets fewer rivulets are present based on the dynamics of the local wettability.

Lastly, it should be mentioned that this thesis is of cumulative nature.
It encompasses the work of four peer review publications.
The first three chapters are intended as a general introduction to the topic, the theory and the numerical method.
Introducing, therefore, the concept of a thin liquid film, the governing equations of motion and suitable numerical algorithms to solve them. 
Following these introductory chapters are the four chapters containing the four peer reviewed articles and one chapter to conclude the thesis.

\newpage
\section*{Kurzfassung}
Hydrodynamik, als solches die Bewegung von Flüssigkeiten, ist ein wichtiger Teil unseres Alltags.
Die definierenden Gleichungen beschreiben nicht nur die Dynamik von Wettersystemen ($10^6$m), sondern auch die Benetzung fein strukturierter Oberflächen ($10^{-9}$m).
Es ist erstaunlich, dass die Theorie der Hydrodynamik sowohl die Bewegung von atmosphärischen Schichten als auch die Verschiebung der Kontaktline eines Tropfens mit einer Größe kleiner als ein Mikrometer ($10^{-6}$m) beschreiben kann.
Diese Eigenschaft der Skaleninvarianz macht die Hydrodynamik zu einem äußerst interessanten Forschungsgebiet.

Die Frage der Schmiermittelreibung besch\"aftigte Osborn Reynolds bereits im 19. Jahrhundert.
Dabei stellte er fest, dass ein Schmiermittel einen dünnen Film zwischen den mechanischen Komponenten bildet.
Die Tatsache, dass der Film \"ublicherweise sehr d\"unn ist, machte es Reynolds möglich vereinfachte Gleichungen aus der Navier-Stokes Gleichung abzuleiten.
Diese Lubrikationsapproximation fand ihre Anwendung in der Dünnfilmgleichung, die seit mehr als 100 Jahren neues Licht auf diese vermeintlich einfachen Probleme wirft.

Heutzutage finden sich viele Anwendungsf\"alle dieser Gleichung, zum einen, um Ph\"anomene in der Natur zu studieren, oder aber auch, um industrielle Prozesse zu optimieren.
Insekten würden zum Beispiel in einem Wassertropfen ertrinken, wenn sie von einem Tropfen benetzt würden. 
Die Natur oder besser gesagt die Evolution l\"oste diese Problem indem Insektenpanzer nur schwer benetzbar oder einfacher gesagt ``wasserabweisend'' sind.
In der Landwirtschaft wiederum sollten Pestizide an den Pflanzen haften und nicht einfach im Boden versickern.
Ob dies wirklich der Fall ist, hängt unter anderem, stark von den Benetzungseigenschaften der Trägerflüssigkeit des Pestizids ab. 
Um eine bessere Benetzung zu garantieren, wird mit oberflächenspannungsreduzierenden Komponenten, sogenannten Tensiden, gearbeitet.
Ein anderer industrieller Sektor, der auch in Zukunft noch wichtiger werden wird, beschäftigt sich mit der Herstellung von Computerchips.
Der Prozess vom Silizium Einkristall zum logischen Schaltkreis (Chip) ist hochkomplex. 
Einer der Produktionschritte ist das ``Drucken'' einer Struktur per Lithographie Verfahren. 
Dazu wird aus einem Tropfen per Rotationsbeschichtung ein Film aufgetragen, jede kleine Unregelm\"a{\ss}igkeit dieses Films kann zu Sch\"aden am Endprodukt f\"uhren. 
Was diesen Problemen gemein ist, ist die hervorragende Rolle der Benetzbarkeit und die Dynamik des d\"unnen Films.

In den letzten sechzig Jahren sind numerische Methoden zur Untersuchung von hydrodynamischen Problemen immer wichtiger geworden.
Kulminiert ist dieser Trend in einem eigenen Forschungsfeld, der numerischen Strömungsmechanik, oder computational fluid dynamics (CFD).
Aber auch außerhalb der Forschung findet CFD ein breites Anwendungsspektrum.
Zum einen werden CFD Simulationen genutzt um Flugzeugflügel oder Motoren von Kraftfahrzeugen zu optimieren, zum anderen findet CFD auch Anwendung in Filmen und Computerspielen.
Unter dem zahlreichen Methoden der numerischen Strömungsmechanik findet sich auch die Gitter Boltzmann Methode (LBM), die sich im Gegensatz zu vielen anderen Methoden der Boltzmann Gleichung der Statistischen Physik bedient.
Per se l\"asst sich die LBM aber nicht als L\"oser f\"ur die D\"unnfilmgleichung nutzen, dazu bedarf es einiger Annahmen und Erweiterungen.
Ausgangspunkt für unser Modell ist die LBM für seichte Gewässer (shallow waters). 
Mit ein paar Annahmen und etwas Algebra l\"asst sich das Gleichungssystem des seichten Gewässers in die D\"unnfilmgleichung \"uberf\"uhren.
Das so entstandene Modell wird mit bekannten Problem der Dünnfilmgleichung getestet und validiert.
Diese ``benchmark''-Probleme reichen von Instabilitäten über Tropfendynamik bis hin zur Entnetzung von Oberflächen.

Der Begriff Entnetzung bedeutet nichts anderes als das Aufbrechen eines Flüssigkeitsfilms, es entsteht also ein Loch.
Anstelle einer homogenen Flüssigkeitsschicht wachsen die L\"ocher und es entstehen einzelne Tropfen. 
Im Laufe dieser Arbeit wird Entnetzung ein immer wiederkehrendes Thema sein.
Es wird sich zum Beispiel die Frage stellen welche Rolle thermischen Fluktuationen bei der Entnetzung von d\"unnen Filme spielen.
Der hier definierte Ansatz macht es möglich, mit einer simplen Erweiterungen deren Effekt zu simulieren und zu klassifizieren.
Die somit gewonnen Erkenntnisse können vorherige Studien reproduzieren und teilweise ergänzen.

Entnetzung von Flüssigkeiten kann viele Gründe haben, zum Beispiel durch räumliche und zeitliche Änderungen der Benetzbarkeit der Oberfläche.
Diese Dynamik zu verstehen ist in keinster Weise trivial. 
Die Deutsche Forschungsgemeinschaft (DFG) hat deswege Ende 2019 ein Schwerpunktsprogramm gefördert, welches sich unter anderem mit dieser Fragestellung widmet. 
Konkret besch\"aftigt sich das SPP 2171 mit dem dynamischen Benetzen von flexiblen, adaptiven and schaltbaren Substraten.
Deswegen findet sich in dieser Arbeit auch der Themenkomplex der schaltbaren Substrate, also Oberflächen, die ihre Benetzungseigenschaft durch zum Beispiel, Beleuchtung \"andern.
Wir entwickeln daf\"ur ein einfaches Modell eines schaltbaren Substrates und zeigen, dass es damit es möglich ist die Entnetzungsmorphologie des dünnen Films zu ändern.
Die Kopplung des schaltbaren Substrates mit der Entnetzung des Films erzeugt, f\"ur gewisse Parameter, einen metastabilien Bandzustand anstelle einzelner Tropfen.
Die Fl\"ussigkeit sammelt sich nach dem Aufbrechen des Films nicht in einzelnen Tropfen, sondern in mehrere Rinnsale.
Ein treibender Effekt dieser Morphologie\"anderung ist die Oberfl\"achenspannung und die lokale Benetzbarkeit.

Abschlie{\ss}end handelt es bei dieser Dissertation um eine kumulative Dissertation.
Den Gro{\ss}teil der Arbeit machen vier Publikation aus Peer Review Journalen aus.
Die ersten drei Kapitel dienen als Einf\"uhrung in das Gebiet, die mathematischen Gleichungen und die numerische Methode.
Wir besch\"aftigen uns daher in diesen einf\"uhrenden Kaptilen mit d\"unnen Filmen, der Navier-Stokes Gleichung und mit verschidenen Algorithmen um die D\"unnfilmgleichung numerisch zu l\"osen.
Danach folgen vier Kapitel mit den Publikation und ein Kapitel in dem die gesamte Arbeit zusammengefasst wird. 

\end{abstract}
