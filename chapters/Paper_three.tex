\chapter{Controlling the dewetting morphologies of thin liquid films by switchable substrates}
\label{chapter:third_paper}
\epigraph{\textit{Du wirst nicht enttäuscht, wenn du nie etwas erwartest und bevor du etwas falsch machst, dann mach mal lieber gar nichts.}}{Kraftklub}

\textit{\small{This chapter has been published as: Zitz, S., Scagliarini, A., Harting, J. (2023). Controlling the dewetting morphologies of thin liquid films by switchable substrates. Physical Review Fluids, \textbf{8}, L122001}}\\

Switchable and adaptive substrates emerged as valuable tools for controlling wetting and actuation of droplet motion. 
Here, we report a computational study of the dynamics of an unstable thin liquid film deposited on a switchable substrate, modeled with a space- and time-varying contact angle. 
For a sufficiently large rate of wettability variation, a topological transition appears. 
Instead of breaking up into droplets, as expected for a substrate with multiple wetting minima, a metastable rivulet state emerges. 
A criterion discriminating whether or not rivulets occur is identified in terms of a single dimensionless parameter. 
Finally, we show and derive theoretically how the film rupture times, droplet shape, and rivulet lifetime depend on the pattern wavelength and speed.
\\
\newcommand{\ts}{\textsuperscript}

\section{Introduction} 
Wet surfaces and droplets are part of our everyday experience and of numerous industrial processes including coating, tribology, painting, and printing, to name but a few~\cite{grossFluidFilmLubrication1980,szeriFluidFilmLubrication2010,quereFluidCoatingFiber1999,dasilvasobrinhoStudyDefectsUltrathin1999,singhInkjetPrintingProcess2010,wijshoffDynamicsPiezoInkjet2010}. 
Moreover, the continuously growing interest for lab-on-a-chip devices~\cite{samieiReviewDigitalMicrofluidics2016,fockeLabonaFoilMicrofluidicsThin2010} as well as for printable electronics or printable photovoltaics~\cite{brabecPlasticSolarCells2001,ronsinRoleInterplaySpinodal2020}, whose efficiency relies crucially on a precise control of material deposition upon (de)wetting of liquid films, drew the attention to applications where the substrate is adaptive or switchable, i.e., it is not inert but responds dynamically to external stimuli or to the evolution of the coating liquid film itself~\cite{buttAdaptiveWettingAdaptation2018}. 
Several realizations of switchable and adaptive substrates have been proposed~\cite{xinReversiblySwitchableWettability2010}, involving smart materials such as polymer brushes~\cite{stuartEmergingApplicationsStimuliresponsive2010}, thermal-responsive hydrogels~\cite{chenThermalresponsiveHydrogelSurface2010}, light-responsive molecules and microstructures~\cite{ichimuraLightDrivenMotionLiquids2000}, or processes such as electrowetting~\cite{mugeleElectrowettingConvenientWay2005}. 
Modeling the coupled problem of a thin film on a switchable substrate and as such the dynamics of the substrate as well as the film is far from being trivial but can be realized by a space- and time-dependent wettability pattern~\cite{grawitterSteeringDropletsSubstrates2021}.
While a consistent body of theoretical/computational work has been devoted to processes on static heterogeneous substrates, the time-dependent case is still almost unexplored, with few relevant exceptions focusing on single droplet spreading and sliding~\cite{grawitterSteeringDropletsSubstrates2021,grawitterDropletsSubstratesOscillating2021,thieleGradientDynamicsModel2020} or limited to analyzing the linear regime~\cite{sumanDynamicsThinLiquid2006}.

Here, we study by means of numerical simulations, the full dewetting dynamics of a thin liquid film deposited on a substrate with a time-varying wettability pattern, starting from a slightly perturbed film to film rupture and long-time morphology~\cite{konnurInstabilityMorphologyThin2000}. 
We identify two regimes where the rupture times grow with the pattern wavelength either linearly (on a static pattern) or attain a constant value (in the time-dependent case), for short wavelengths, and approach a quadratic law as the wavelength increases. 
These observations are then explained theoretically. 
We show that, by tuning the rate of change of the underlying pattern, one can control the dewetting morphology, which is often desired in microfluidic applications.
In particular, for large enough pattern speeds, we detect a state where the film retracts into metastable rivulets, eventually breaking up into multiple droplets.
We introduce a control parameter to discriminate whether rivulets or just droplets (as in the static situation) can be observed and propose a phenomenological argument to justify the logarithmic dependence of the rivulets' lifetime on the pattern speed.

\section{Method} 
In order to simulate the dewetting dynamics on patterned, ``switchable,'' substrates, we integrate numerically the thin-film equation~\cite{oronLongscaleEvolutionThin1997,crasterDynamicsStabilityThin2009} 
\begin{equation}\label{eq:thinfilm}
    \partial_t h(\mathbf{x},t) = \boldsymbol{\nabla}\cdot\left[M_{\delta}(h)\boldsymbol{\nabla} p(\mathbf{x},t)\right],
\end{equation}
by means of a recently developed lattice Boltzmann (LB) scheme~\cite{zitzLatticeBoltzmannMethod2019,zitzLatticeBoltzmannSimulations2021, zitzSwalbeJlLattice2022}.
Equation~(\ref{eq:thinfilm}) describes, in a lubrication approximation spirit, the evolution of the height field (film thickness) $h(\mathbf{x},t)$, denoting the location of the liquid/gas interface. 
The mobility function $M_{\delta}(h) = \frac{2h^3 + 6\delta h^2 + 3\delta^2 h}{6\mu}$ depends on the velocity boundary condition at the substrate, parametrized by an effective slip length $\delta$ (for $\delta \rightarrow 0$ it reduces to the no-slip form $h^3/(3\mu)$)~\cite{huhHydrodynamicModelSteady1971, peschkaSignaturesSlipDewetting2019, fetzerQuantifyingHydrodynamicSlip2007, munchLubricationModelsSmall2005}. 
Here, $\mu$ is the fluid dynamic viscosity.
The film pressure $p(\mathbf{x},t)$ consists of the sum of the Laplace and disjoining pressures, that is, $p(\mathbf{x},t) = -\gamma \nabla^2 h - \Pi$ ($\gamma$ is the surface tension~\cite{wuHowChemicalPatterns2020}).
The disjoining pressure $\Pi$ can be seen as (minus) the derivative, with respect to the film thickness, of an effective interfacial potential~\cite{DeryaguinChuraev1978}.
As such, it contains the information on the liquid/solid and solid/gas interactions and, hence, on the wettability, which is parametrized in terms of the contact angle $\theta$~\cite{bonnWettingSpreading2009, schwartzSimulationDropletMotion1998, youngIIIEssayCohesion1805, degennesWettingStaticsDynamics1985}. 
The expression adopted for $\Pi$ is
\begin{equation}\label{eq:disjoinpressure}
\Pi(h,\theta) = \frac{2\gamma}{h_{\ast}}\{1-\cos[\theta(\mathbf{x},t)]\}
  f\left(\frac{h}{h_{\ast}}\right),
\end{equation}
where $f(\xi)=\xi^{-3} - \xi^{-2}$. 
$h_{\ast}$ is the height at which $\Pi$ vanishes and sets the precursor layer thickness~\cite{schwartzSimulationDropletMotion1998, mitlinDewettingSolidSurface1993, teletzkeHowLiquidsSpread1987}.
The time variation of the patterned substrate enters the model through the disjoining pressure, by making the contact angle space and time dependent, i.e.,~$\theta = \theta(\mathbf{x},t)$.
We decorate the substrate with a checkerboard pattern, a common choice that generalizes the broken homogeneity of the stripes to two directions~\cite{jalaliFabricationCharacterizationScalable2018, nagayamaIntermediateWettingState2020, dasSurfacedirectedSpinodalDecomposition2020}.
In particular, we employ the sinusoidal form
\begin{equation}\label{eq:sinetheta}
   \!\! \theta(\mathbf{x},t) = \theta_0 + \delta\theta\left\{\sin\left[q_{\theta} (x+v_{\theta x}t)\right]\sin\left[q_{\theta}(y+v_{\theta y}t)\right]\right\},\! 
\end{equation}
where $q_{\theta} = 2\pi/\lambda$, i.e., the pattern evolves in time as a plane wave.
We fix the velocity direction to one diagonal, namely $\mathbf{v}_{\theta} = (v_{\theta x},v_{\theta y}) = v_{\theta}(1/\sqrt{2},-1/\sqrt{2})$ (we will return later to the importance of this choice), and we set $\theta_0 = 20^{\circ}$ and $\delta\theta=10^{\circ}$~\footnote{Since a typical velocity is such that $v_{\theta} \Delta t \ll \Delta x$ (in one time step $\Delta t$ the wave would travel a distance much smaller than a lattice spacing $\Delta x$),the time update needs to be interpreted in an integer part sense, that is, the pattern is shifted by $\Delta x$ every $1/v_{\theta x}$ $\Delta t$ (equivalently in $y$-direction).}.
Length scales and timescales will be expressed, respectively, in units of the mean film height, $h_0$ (which is constant in time, due to mass conservation), and of $t_0 = \frac{3\mu}{\gamma h_0^3 q_0^4}$, the inverse growth rate of the most unstable mode, whose wave number is $q_0$, of a spinodally dewetting film~\cite{meckeThermalFluctuationsThin2005}. 
On a uniform substrate, with constant contact angle $\theta^{(u)}$, the wave number reads $(q^{(u)}_0)^2 = h_{\ast}^{-2}(1-\cos \theta^{(u)})f^{\prime}(h_0/h_{\ast})$~\cite{meckeThermalFluctuationsThin2005, zhangMolecularSimulationThin2019}. 
In our patterned case, we define $q_0^2=h_{\ast}^{-2}(1-\cos\theta_0)f^{\prime}(h_0/h_{\ast})$. 
Correspondingly, we choose as a velocity scale $v_0 = \lambda_s/t_0$, where $\lambda_s = 2\pi/q_0$.
\begin{figure}
    \centering
    \includegraphics[width=0.75\textwidth]{graphics/Figure_1.png}
    \caption{Stationary film thickness field ($t>t_0$) showing the formation of droplets. The color map indicates the contact angle pattern (Eq.~(\ref{eq:sinetheta}) with $v_{\theta}=0$), with lower (higher) values in light blue (yellow).
    }
    \label{fig:handtheta}
\end{figure}
%%%%
Figure~\ref{fig:handtheta} shows $h(\mathbf{x},t)$ (droplets) and $\theta(\mathbf{x})$ (color coded) for $v_{\theta} = 0$ (i.e., the static case) and $\lambda = 256 h_0$~\footnote{For a better visualization double the domain length $L$ and periodically continue the image.}, in the late stages of dewetting.
As expected, droplets form in regions of small contact angles (blue) while the regions of high contact angles (yellow) dewet.

\section{Results} We first investigate how the rupture times depend on the parameters characterizing the wettability pattern, namely the wavelength of the contact angle variation $\lambda$ and wave speed $v_{\theta}$~\cite{karguptaMorphologicalSelforganizationDewetting2002,karguptaInstabilityPatternFormation2000,nisatoExcitationSurfaceDeformation1999,karimPhaseSeparationUltrathin1998, succiLatticeBoltzmannEquation2001}.
The film rupture time $\tau_r$ is defined as the least $t$ such that $h(\mathbf{x},\tau_r)=h_{\ast}$ (i.e., when the free surface ``touches'' the substrate).
In Fig.~\ref{fig:model_rt} we report the rupture times as a function of the wavelength, for stationary ($v_{\theta}=0$) and time-dependent ($v_{\theta}=20 v_0$) patterns. 
It is conveyed that, overall, rupture occurs earlier on the static substrate, suggesting that the time variation tends to stabilize the film, in agreement with linear stability analysis results~\cite{sumanDynamicsThinLiquid2006}.
We observe that $\tau_r$ grows linearly with $\lambda$ for short wavelengths and quadratically for longer $\lambda$.
These facts can be qualitatively explained as follows. 
In this case, from the linearized thin-film equation (in one spatial dimension, for simplicity), obtained setting $h=h_0 + \delta h$ with $\delta h \ll h_0$, we can easily see that the exponential growth of the height perturbation is affected by the wettability pattern (variable contact angle) in such a way that $\partial_t (\delta h) \propto \{\partial_x^2 [\partial_h\Pi(h_0)]\} \delta h$. Therefore, since the characteristic time $t_{\theta}$ can be estimated dimensionally as $t_{\theta} \sim \delta h/\dot{(\delta h)}$, the rupture time should go as
\begin{equation}\label{eq:taur_l2}
    \tau_r \sim t_{\theta} \sim  \delta h/\dot{(\delta h)} \propto \frac{3\mu}{h_0^3}\{\partial_x^2 [\partial_h\Pi (h_0)]\}^{-1} \sim t_0 \left(\frac{q_{\theta}}{q_0}\right)^{-2} \propto t_0 q_0^2 \lambda^2.
\end{equation}
Conversely, for fast growths ($t_{\theta} \ll t_R$), retraction dominates and fixes the timescale, $\tau_r \sim t_R$. 
The latter is related to the time the liquid takes to flow out of regions of high contact angle, whose size is $\sim \lambda$. 
Hence we have 
\begin{equation}\label{eq:taur_l1}
 \tau_r \sim \tau_R \propto U_{\Theta}^{-1}\lambda,
\end{equation}
where $U_{\Theta}$ is the retraction speed $U_{\Theta} = \frac{\gamma \Theta^3}{9\mu}$~\cite{edwardsNotSpreadingReverse2016}, with $\Theta = \max_{\mathbf{x}}\{\theta(\mathbf{x})\}$.
\begin{figure}
    \centering
    \includegraphics[width=0.4\textwidth]{graphics/Figure_2.pdf}
    \caption{Rupture times $\tau_r$ as a function of the pattern wavelength $\lambda$, for $v_{\theta}=0$ (\textcolor{jlblue}{$\bullet$}) and $v_{\theta}=20 v_0$ (\textcolor{jlorange}{$\star$}).
    The continuous and dashed lines indicate the linear, $\sim \lambda$, and quadratic, $\sim \lambda^2$, scaling laws, respectively.
        }
    \label{fig:model_rt}
\end{figure}
We now focus on the long-time dynamics, the characterization of the dewetting morphologies, and how they are affected by the speed of the wettability wave.
\begin{figure}
    \centering
    \includegraphics[width=0.4\textwidth]{graphics/Figure_3.pdf}
    \caption{Main panel: Time evolution of the height fluctuations $\Delta h(t)$ during the dewetting process on the patterned substrate given by Eq.~(\ref{eq:sinetheta}) with $v_{\theta}= 0$ and $\lambda= 512 h_0$ (\textcolor{jlblue}{$\bullet$}), $\lambda=256 h_0$ (\textcolor{jlorange}{$\blacksquare$}) and $\lambda=170 h_0$ (\textcolor{jlgreen}{$\star$}). 
    Inset: Number of droplets $N(t)$ as a function of time. 
    The three horizontal dashed lines indicate the number of minima of Eq.~(\ref{eq:sinetheta}), which is $2\left(\frac{L}{\lambda}\right)^2$. 
    The snapshots depict the stationary droplet states as greyscale images of the film thickness field $h(\mathbf{x},t)$.
      }
    \label{fig:clusters_v0_sine}
\end{figure}
On the stationary substrate, after rupture all fluid accumulates in droplets centered at contact angle minima.
Consequently, as seen from the inset of Fig.~\ref{fig:clusters_v0_sine}, where we plot the number of droplets $N(t)$ versus time~\footnote{A droplet is identified by the set (``cluster'') of points, in the plane, constituting each of the connected components of the set $\{\mathbf{x} \in [0,L]^2 | h(\mathbf{x},t) \geq h_{\ast}$\}.}, in the steady state ($t \gg t_0$) $N(t)$ attains the value $N_{\infty} = 2(L/\lambda)^2$ (horizontal lines), which equals the minima of Eq.~(\ref{eq:sinetheta}), for $v_{\theta}=0$.
Notice that the number of droplets converges faster for smaller pattern wavelengths, in line with the observation reported and justified in the previous section that the characteristic dewetting time decreases with the wavelength.

In the main panel of the figure, the height fluctuations $\Delta h(t) = \max_{\mathbf{x}}\{h(\mathbf{x},t)\}-\min_{\mathbf{x}}\{h(\mathbf{x},t)\}$ grow in time until film rupture and then settle to a constant value. 
This represents a measure of the mean droplet height $h_d$ (since droplets are essentially monodisperse), decreasing with the pattern wavelength (as expected, due to a decreasing droplet volume, $V_d = \frac{h_0 \lambda^2}{2}$).
\begin{figure}
    \centering
    \includegraphics[width=0.4\textwidth]{graphics/Figure_4.pdf}
    \caption{Time evolution of the second-order Minkowski structure metric $q_2(t)$ for different $\Gamma$ values, on a substrate with pattern wavelength $\lambda=256 h_0$.
    The grey-scale insets supply snapshots of the corresponding film thickness fields.}
    \label{fig:msm_q2}
\end{figure}

A time-dependent pattern affects the dewetting morphology quite substantially.
For $v_{\theta} = 1.7 v_0$ we still observe the formation of droplets, similarly to the stationary case ($v_{\theta} = 0$). 
However, these are transported with the contact angle minima, reproducing a somehow similar behavior recently described in a numerical study of a droplet on a moving wettability step~\cite{grawitterSteeringDropletsSubstrates2021}.
If the pattern speed is further increased, for $v_{\theta} = 17.5 v_0$ we observe the development of rivuletlike structures, aligned with $\mathbf{v}_{\theta}$. 
The film, in fact, while dewetting in the direction normal to the pattern velocity, is exposed, in the direction of the velocity, to a periodic potential with alternating minima and saddle points, which partially (as we will see) stabilizes the film over ``preferential'' lanes along the diagonals. 
This makes the chosen velocity direction, $(1/\sqrt{2},-1/\sqrt{2})$ (or, equivalently, the orthogonal one $(1/\sqrt{2},1/\sqrt{2})$), optimal for the formation of rivulets. 

In order to better characterize the various morphologies we apply the theory of Minkowski's functionals. 
In particular, we employ the second-order Minkowski structure metric $q_2$~\cite{mickelShortcomingsBondOrientational2013, schallerPapaya22DIrreducible2020}, which can be computed from a Voronoi tessellation of the set of discrete points $(x_i, y_i)$ on the two-dimensional (2D) lattice, such that the height field lies above a certain threshold\footnote{The expression is $q_2 = \frac{1}{N}\sum_j \frac{1}{P_j}|\sum_k L^{(j)}_k e^{2i\phi^{(j)}_k}|$, where the inner sum runs over the edges of length $L^{(j)}_k$, of the $j$th Voronoi cell, whose perimeter is $P_j$, and $\phi^{(j)}_k$ is the polar angle of the normal to the $k$th edge. 
The outer sum represents an ensemble average over the $N$ points in the set.}. The $q_2$ metric quantifies the degree of anisotropy of the dewetting morphology, so it takes relatively large values if the structures formed display a preferential direction. 
Measuring $q_2$ then enables us to clearly distinguish between the formation of droplets and rivulets: Much larger $q_2$ values are attained for the latter type of structure, as we can see in Fig.~\ref{fig:msm_q2}.
We observe, on the other hand, that such rivulets are metastable and eventually break up into droplets, as indicated by the collapse of $q_2$ at later times. 
Notice, though, that the $q_2$ signal for any $v_{\theta} >0$ always stays above the one for the static case, suggesting that even the smallest pattern velocity introduces a sizeable deformation of the spherical cap shape.
\begin{figure}
    \centering
    \includegraphics[width=0.4\textwidth]{graphics/Figure_5.pdf}
    \caption{Main panel: Rivulet lifetimes $\tau_{\text{riv}}$ for various $\Gamma$.
    The dashed line is a guide to the eye to highlight the logarithmic dependence, in agreement with the theoretical prediction, Eq.~(\ref{eq:rivlt}).
    Inset: Height fluctuations $\Delta h(t)$ vs time, along the rivulet axis, for three different $\Gamma$.
    }
    \label{fig:stab_ligs_lam2}
\end{figure}
The breakup is the result of a varicose mode of the rivulet~\cite{diezBreakupFluidRivulets2009, mechkovStabilityLiquidRidges2008}, whose wavelength is $\approx\lambda$, such that only $N_{\infty}/2$ droplets are counted after breakup. 
These droplets show a peculiar dynamics, characterized by a periodic sequence of spreading and retraction, driven by the pattern, that we dub "pumping state"~\footnote{see movie ligament\_formation\_and\_breakup.mp4 in~\ref{suppmat}.}.

We argue that the emergence of rivulets is controlled by the competition of two characteristic velocities, the pattern wave speed $v_{\theta}$, and the retraction speed $U_{\Theta}$, introduced in Eq.~(\ref{eq:taur_l1}). 
If $U_{\Theta}$ is large as compared to $v_{\theta}$, the film retraction is faster than the local contact angle variation and thus droplets form. 
However, if $v_{\theta}$ is larger than $U_{\Theta}$, then the retracting film has too little time to form droplets and ends up in the metastable rivulet state. 
It appears, therefore, natural to consider the ratio of these two velocities, $\Gamma \equiv v_{\theta}/U_{\theta}$, as the discriminating parameter.
We see from Fig.~\ref{fig:msm_q2} that indeed rivulets form only for $\Gamma > 1$. 
Moreover, the larger $\Gamma$, the more stable are the rivulets; in other words, the rivulet lifetime $\tau_{\text{riv}}$, that can be conventionally taken as the time at which the drop of $q_2$ occurs, grows with $v_{\theta}$ (see Fig.~\ref{fig:stab_ligs_lam2}).
The rivulet itself is, in fact, prone to dewetting, with the liquid accumulating over patches around contact angle minima. 
However, as the pattern moves, the instability is tamed due to configurations whereby higher contact angle regions underlie height field maxima, thus tending to revert the fluid flow.
Heuristically speaking, this means that, if we evaluate $\Delta h(t)$ restricted on the rivulet axis, it should grow exponentially (with a certain growth rate $\alpha \propto t_0$) only when the system is in the unstable configuration. 
Namely, $\Delta h(t)/\Delta h_0 \propto e^{\alpha t}$ (see inset of Fig.~\ref{fig:stab_ligs_lam2}) with a prefactor proportional to the time spent by the rivulet in such a configuration, which goes as $\sim \lambda/v_{\theta}$, therefore $\Delta h(t)/\Delta h_0 \sim \alpha (\lambda/v_{\theta})e^{\alpha t}$. 
The rivulet lifetime can be seen as the rupture time of the structure along its axis, hence such that $\Delta h (\tau_{\text{riv}}) \sim h_0$~\cite{zitzLatticeBoltzmannSimulations2021}, which yields
\begin{equation}\label{eq:rivlt}
    \tau_{\text{riv}} \sim \alpha \log(v_{\theta}) \propto t_0 \log(\Gamma). 
\end{equation}
This logarithmic dependence is indeed observed in the numerical data as shown in Fig.~\ref{fig:stab_ligs_lam2}.

We envisage a possible realization of a dewetting experiment on a switchable substrate of the type modeled by the spatiotemporal contact angle (\ref{eq:sinetheta}). 
One may think of a thin liquid film cast on a light responsive substrate~\cite{ichimuraLightDrivenMotionLiquids2000}, under the action of controlled external stimuli (a light emitter). 
An ideal candidate could be a digital multimirror device (DMD).
This technology was effectively used for thin-film experiments and additive manufacturing~\cite{vieyrasalasActiveControlEvaporative2012, sahaScalableSubmicrometerAdditive2019}. 
It allows for fast temporal modulations of the optical signal (with frequencies up to $\approx 16 \, \text{kHz}$) with spatial resolution of $\approx 10 \times 10 \, \mu \text{m}^2$ (the size of a pixel).
Considering as a reference, for instance, the system studied in Refs~\cite{beckerComplexDewettingScenarios2003,fetzerThermalNoiseInfluences2007}, namely a $\sim 4-\text{nm}$-thick film of polystyrene deposited on an oxidized silicon wafer, we evaluate the retraction speed $U_{\Theta} = \frac{\Theta^3 \gamma}{9 \mu}$ to be $U_{\Theta} \approx 10^{-2} \, \mu \text{m}/\text{s}$. 
The (minimum) \textit{pattern speed} can be estimated from the pixel size with frame rate $\sim 1 s^{-1}$ as $v_{\theta} \sim 10 \, \mu \text{m}/\text{s}$, which would result in $\Gamma \sim 10^3$, i.e., well within the rivulet regime ($\Gamma > 1$). 
Also, both $U_{\Theta}$ and $v_{\theta}$ can be widely modulated, the former by varying the (temperature- and molecular-weight-dependent) viscosity or tailoring the substrate to make it more hydrophobic (i.e., increasing the contact angle), and the latter by tuning the spatial resolution and frame rate of the DMD. 
Thus, we expect the range of achievable $\Gamma$'s to be feasibly extended both to very high ($\Gamma \gg 1$) and very low ($\Gamma \ll 1$) values.

\section{Conclusions}
We presented numerical simulations and a theoretical analysis on the dewetting of thin liquid films on a switchable substrate.
Studying how the film stability depends on the underlying static pattern, we found that the rupture times grow linearly with the pattern wavelength, for short wavelengths, and quadratically in the long-wavelength limit. 
In the time-dependent case, the rupture times are generally longer, indicating an induced greater film stability, and, while the quadratic growth is preserved at long wavelengths, a plateauing behavior was observed as the wavelength decreases. 
Furthermore, we showed that, at increasing the wettability wave speed, a transition occurs in the dewetting morphology from a multidroplet to a metastable multirivulet state. 
We find that this surprising morphological transition can be described with a single dimensionless parameter $\Gamma$.
Considering only the ratio of the pattern speed and the typical film retraction speed, the rivulets' lifetime itself grows with the pattern speed, displaying a logarithmic dependence that was captured by means of phenomenological arguments.
On a broader perspective, our work suggests that switchable substrates offer another avenue to control thin-film dewetting, with obviously relevant implications, for instance, for open microfluidic devices, and paves the way to future studies in this direction, exploiting more complex and dedicated space-time dependencies.

\section{Acknowledgements} We acknowledge financial support from the German Research Foundation (DFG) (priority program SPP2171 / project HA-4382/11 and Project-ID 431791331—CRC1452), and from the Independent Research Fund Denmark (grant 9063-00018B).
\newpage

\section{Supplemental material}.
\label{suppmat}
\subsection{Minkowski's structure metric \texorpdfstring{$q_2$}{hmm} in an extended parameter space.}
\begin{figure}
    \centering
    \includegraphics[width=0.4\textwidth]{graphics/SupMatFig_1.pdf}
    \includegraphics[width=0.4\textwidth]{graphics/SupMatFig_2.pdf}
    \caption{LEFT PANEL. Minkowski's structure metric $q_2$ for three different wavelengths $\lambda=64 h_0$, $\lambda=128 h_0$ and $\lambda=256 h_0$ (as in figure 4 of the main text) and for $\Gamma=1.5$ (all other parameters are as in the main text). 
    The time interval during which $q_2 \approx 1$ signals the emergence of rivulets, also for $\lambda = 64 h_0$ which is comparable with the spinodal wavelength $\lambda_s \approx 70 h_0$. 
    RIGHT PANEL. Comparison of the Minkowski's structure metric $q_2$ for $\delta\theta=5^{\circ}$ and $\delta\theta=10^{\circ}$, $\Gamma = 15$ and $\lambda = 256 h_0$.}
    \label{fig:q2_difflambda}
\end{figure}
\noindent In this section we test the robustness of the observation of the rivulet state over a wider parameter space and in particular at changing: 1) the 
pattern wavelength $\lambda$ and 2) the heterogeneity amplitude $\delta \theta$.
To this aim we have run simulations with $\Gamma=1.5$, $\lambda/h_0 = 64, 128$, and with $\Gamma = 15$, $\lambda = 256 h_0$, $\delta \theta = 5^{\circ}$, respectively.
In Fig.~\ref{fig:q2_difflambda} (left panel) we report the measurements of the Minkowski's structure metric $q_2$, whose increase from zero up to a plateauing value of $q_2 \approx 1$ indicates the emergence of rivulets, for the different $\lambda$'s ($\lambda=256 h_0$, which is the value considered in the main text, is also reported for comparison).
Analogously, in the right panel, we report $q_2$ as a function of time for $\delta \theta = 5^{\circ}$  and $\delta \theta = 10^{\circ}$ (the value used in the main text). 
We see that the $q_2$ metric attains the value $q_2 \approx 1$, signalling the rivulet state, albeit over a time interval shorter than for $\delta \theta = 10^{\circ}$, i.e. the rivulets lifetime decreases with $\delta \theta$. 
This was somehow expected, since obviously (and also in the static case) the patterning looses effectiveness as the contact angle mismatch is reduced (see, e.g.~\cite{konnurInstabilityMorphologyThin2000}).

\subsection{Droplet shape}

\noindent We investigate, here, how the local contact angle of droplets, formed on the more hydrophilic patches after dewetting, depends on the pattern wavelength. 
We recall, in fact, that the patterning is such that the contact angle is not piecewise constant, but varies with continuity.
The droplet shape is determined by minimization of the total interfacial energy
\begin{equation}\label{eq:energy}
  E = \gamma_{\text{lg}} A_{\text{lg}} + \int_{A_{\text{sl}}} (\gamma_{\text{sl}} - \gamma_{\text{sg}})d \sigma,
\end{equation}
where $A_{\text{lg}}$ and $A_{\text{sl}}$ are the liquid/gas and solid/liquid interface areas, and $\gamma_{\text{lg}}$, $\gamma_{\text{sl}}$, $\gamma_{\text{sg}}$ are the liquid/gas, solid/liquid and solid/gas interface energies per unit area \cite{wuHowChemicalPatterns2020}. 
In particular, $\gamma_{\text{lg}} \equiv \gamma$ is the surface tension.
Setting $A_{\text{lg}} \equiv A$ and $A_{\text{sl}} \equiv S$, by Young's equation $\cos \theta = \frac{\gamma_{\text{sg}} - \gamma_{\text{sl}}}{\gamma}$, Eq.~(\ref{eq:energy}) can be rewritten as
\begin{equation}\label{eq:energyoung}
  \tilde{E} \equiv \frac{E}{\gamma} = A - \int_S \cos \theta \text{d}x \text{d}y.
\end{equation}
Droplets will form around minima of the contact angle pattern
\begin{equation}\label{eq:contact}
  \theta(x,y) = \theta_0 + \delta \theta \sin(q_{\theta} x) \sin (q_{\theta} y) \qquad q_{\theta} = \frac{2\pi}{\lambda},
\end{equation}
namely $(x_n,y_n) = \left((2n+1)\frac{\lambda}{4},(2n+3)\frac{\lambda}{4}\right)$, with $n=0,\pm 1, \pm 2,\dots$.
If we consider large wavelengths ($q_{\theta} h_0 \ll 1$) and heterogeneity ($\delta \theta \ll 1$) such that the contact angle gradients are small, the
expression (\ref{eq:contact}) can be expanded as
\begin{equation}\label{eq:contact2}
  \theta(x,y) \approx \theta_m + \delta \theta q_{\theta}^2 ((x-x_n)^2 + (y - y_n)^2) + o(|\mathbf{x}-\mathbf{x}_n|^2),
\end{equation}
where $\theta_m = \theta_0 - \delta \theta$.
This local radial symmetry allows to approximate the equilibrium droplet shape as a spherical cap of height $h$ and base radius $a$, whose area is $A = \pi(a^2 + h^2)$; inserting the expression for $A$ and (\ref{eq:contact2}), neglecting higher than second order terms, in (\ref{eq:energyoung}) gives
\begin{equation}\label{eq:energy3}
  \tilde{E}(h,a) = \pi(a^2 + h^2) -
  \int_{\mathcal{C}_a(\mathbf{x}_n)} \cos\left[\theta_m + \delta \theta q_{\theta}^2 ((x-x_n)^2 + (y - y_n)^2)\right]
    \text{d}x\text{d}y,
\end{equation}  
where $\mathcal{C}_a(\mathbf{x}_n)=\{(x,y) \in [0, L]^2|(x-x_n)^2 + (y-y_n)^2 \leq a^2\}$ is the circle of centre $\mathbf{x}_n$ and radius $a$. 
Due to global volume conservation and assuming the droplets to be monodisperse, the droplet volume is $V_d = h_0L^2/N_d = h_0 \lambda^2/2$, where $N_d$ is the number of droplets, which equals the number of minima of (\ref{eq:contact}) in the domain $[0, L]^2$, i.e. $N_d = 2(L/\lambda)^2$.
Enforcing the volume of the spherical cap to be equal to $V_d$ relates $h$ and $a$ by
\begin{equation}
  \frac{\pi h}{6}(3a^2 + h^2) \approx \frac{\pi}{2} a^2 h =  \frac{h_0 \lambda^2}{2}    
\end{equation}
in the ``lubrication approximation'' $h \ll a$, whence
\begin{equation}
  h \approx \left(\frac{h_0}{\pi}\right) \left(\frac{\lambda}{a}\right)^2.
\end{equation}  
Inserting the latter expression in (\ref{eq:energy3}) and performing the integral, the energy (that we indicate now as $E(a)$ to lighten the notation) reads
\begin{equation}\label{eq:energy4}
  E(a) = \pi \left(a^2 + \frac{h_0^2 \lambda^4}{\pi^2 a^4} \right) - \frac{2\pi}{\delta \theta q_{\theta}^2}
  \left[\sin\left(\theta_m +\frac{\delta \theta}{2}q_{\theta}^2a^2\right) - \sin \theta_m \right].
\end{equation}  
We expand, then, the sine in the second term up to second order in $\delta \theta$ (such that the energy is first order) and we finally get
\begin{equation}\label{eq:energyfin}
  E(a) \approx  \pi \left((1-\cos \theta_m) a^2 + \frac{h_0^2 \lambda^4}{\pi^2 a^4} \right)
  + \delta \theta \frac{\pi}{4}\sin \theta_m q_{\theta}^2 a^4 \equiv E_0(a) + \delta \theta E_1(a).
\end{equation}  
The minimum condition $\frac{\partial E}{\partial a} = 0$~\footnote{It can be easily checked that $\frac{\partial^2 E}{\partial a^2} > 0$ for $a>0$} yields
\begin{equation}\label{eq:minim}
  2\pi (1-\cos \theta_m)a^6 - 4h_0 \lambda^4 + \delta \theta \pi^2 \sin \theta_m q_{\theta}^2 a^8 =0.
\end{equation}  
The solution of (\ref{eq:minim}) at zero-th order in $\delta \theta$ is
\begin{equation}\label{eq:a0}
  a_0 = \left[\frac{2h_0\lambda^4}{\pi^2(1-\cos \theta_m)}\right]^{1/6}.
\end{equation}
To this order the droplet contact angle $\tan (\theta_d/2) = h/a$ reads
\begin{equation}
\tan \left(\frac{\theta_d}{2}\right) = \frac{h}{a} \approx
\left(\frac{h_0}{\pi}\right)\frac{\lambda^2}{a_0^3} =
\left(\frac{1-\cos \theta_m}{2}\right)^{1/2}\frac{\lambda^2}{(\lambda^{2/3})^3} \approx \frac{\theta_m}{2} \quad \Rightarrow \quad \theta_d \approx \theta_m,
\end{equation}
i.e., it does not depend on $\lambda$. 
We move, then, to the next order. We seek a solution to (\ref{eq:minim}) in the form $a = a^{(0)} + \delta \theta a^{(1)} + \dots$, where $a^{(0)} \equiv a_0$.
At the first order in $\delta \theta$ we get
\begin{equation}
  a^{(1)} = -q_{\theta}^2 a_0^3\frac{\sin \theta_m}{6(1-\cos \theta_m)},
\end{equation}
such that the correction to the droplet contact angle provides
\begin{equation}
  \tan \left(\frac{\theta_d}{2}\right) \approx \frac{h_0 \lambda^2}{\pi a_0^3}
  \left(1+\delta \theta q_{\theta}^2 a_0^2 \frac{\sin \theta_m}{12(1-\cos \theta_m)}\right)
  \quad \Rightarrow \quad \theta_d \approx
  \theta_m \left(1+\frac{\delta \theta}{\theta_m}\frac{b}{\lambda^{2/3}}\right),
\end{equation}  
where $b = \left(\frac{32 h_0^2 \pi^4}{27 \theta_m^5}\right)^{1/3}$, i.e. the droplet contact angle grows at decreasing pattern wavelength as $\theta_d \sim \lambda^{-2/3}$.