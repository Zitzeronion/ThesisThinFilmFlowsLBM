\chapter{Conclusion and outlook}
\label{chapter:conclusion}
\epigraph{\textit{Wenn du dann am Boden bist, weißt du wo du hingehörst\\ Wenn du ganz alleine bist, weißt du dass du es noch so lange sein wirst.}}{Faber}

Thin liquid film flows are a prominent phenomenon that we encounter in our daily life, e.g. the rain droplets on the windows or the water soap film when we wash our hands.
Interestingly most of the time we do not recognise them as such, although they are used in a lot of industrial process and applications as motivated in Chap.~\ref{chapter:intro}.
Vivid examples of thin film flows have been shown in Fig.~\ref{fig:examples_intro}. 
Both the life time and shape of a soap bubble can in fact be explained with the theoretical framework derived in Chap.~\ref{chapter:theory}.

That said the greater goal of this project is to understand the dynamics needed for mass production of cheap organic/perovskite photovoltaic cells (OPV/PSC).
While these cells are still lacking in efficiency as compared to silica based ones they offer great usability features.
The underlying structure does not need to be stiff and as such they can be printed on plastic foil.
Printing on soft and elastic substrates with blends of complex liquids is however a hideous problem, more so from a theoretical point of view.
Instead of searching from the very first step for a solution to the full problem, this thesis addresses one issue of this complex process, which is the film formation.
A step of crucial importance that happens after the coating, be it with a slot die or another method.
Numerical simulations based on the thin film equation may help to understand at least this part of the process.
In the long run the found results should help to improve both the process itself as well as the efficiency of the final product.

To this end a novel approach for modelling the dynamics of thin liquid films has been developed, see Chap.~\ref{chapter:theory}-\ref{chapter:second_paper}.
That said thin film flows have been studied with other models but all of them have shortcomings (the same is of course true for the model presented here).
% More thinking here!
The key question that has not been asked in Chaps.~\ref{chapter:intro}-\ref{chapter:third_paper} is: Is there a need for another method, or are there issues that can not be overcome with present solvers? 
The second part of this question can be answered with a cautions \textit{no}.
Many researchers invested a lot of effort in the development of methods and solvers, problems such as dewetting on patterned substrates can be studied with several other solvers.
The first part of the question is a subjective one, because what do we actually need to numerical approximate the thin film equation?
A well posed finite difference scheme is often enough for simple problems.
However when the problems get complex due to couplings such as fluid structure interactions or phase changes (evaporation) we would prefer to use a method that we understand which is trustworthy and has been thoroughly tested. 
I think this one of the strengths of this method, the transparent nature of the software. 
The code base is available under an open source license and thus is freely available to everyone interested.
The package approach, similar to e.g. Python, Matlab or R, makes it simple to share and collaborate with colleagues on numerical experiments.
Of course there is a long list of thin film problems that have been studied with numerical models which will not benefit from our approach.
Spin coating is one of these problems that are outside the capabilities of our method.  

Apart from the these quality of life/usability features there are few things that make the derived model special and scientifically interesting.
In contrast to various numerical methods used in the literature of thin film dynamics, \textit{Swalbe.jl} is build upon the lattice Boltzmann method as described in Chap~\ref{chapter:method}.
It is therefore not intended as a general tool to approximate differential equations, but only for the hydrodynamics of thin liquid films.
Furthermore the underlying set of equations that are approximated by using the lattice Boltzmann method are the shallow water equations, see Chap.~\ref{chapter:theory}.
While the shallow water equations do in fact describe a different system than the thin film equation it has been shown in Chap.~\ref{chapter:first_paper} that the former can be made to approximate the thin film dynamics.
Evaluation of leading order terms and appropriate forcing does lead to a well-defined matching condition between the lattice Boltzmann shallow water model and the thin film equation.
The model is thoroughly tested against relevant problems in the thin film regime. 
Among them is the relaxation of a sessile droplet on a partially wetting substrate measuring Tanner's law as well as the Cox-Voinov relation.
But also the sliding behaviour of a droplet on an inclined plane, where a linear relation between Bond and Capillary number is theoretically predicted and observed.

Having shown agreement with theory as well as experimental findings new dynamics to the here defined model are introduced in Chaps.~\ref{chapter:second_paper}-\ref{chapter:third_paper}.  
First being the addition of thermal fluctuations.
Although this seems like a singular use case the approach derived can in fact be used to match other dynamics, e.g. surfactants.
These fluctuations lead to a constant excitement of the fluid air interface that is smoothed out by the surface tension, leading to the appearance of thermo-capillary waves.
Their amplitude however is usually so small that they can be neglected in experimental analysis or numerical simulations. 
However having a dewetting thin film it can be assumed that the thickness of the film will be comparable to the amplitude of the fluctuations, at least in regions where the film ruptures.
As shown in Chap.~\ref{chapter:second_paper} these fluctuations can be added to the model using a matching condition for a fluctuating force term. 
The fluctuating force term is than tested against the theory of capillary waves and does show in fact good agreement with the predicted spectrum.
In the following it is shown that the stability difference, as measured in rupture times, between fluctuating and deterministic thin film~\footnote{$k_BT > 0$ fluctuating thin film, $k_BT = 0$ deterministic thin film} does inversely depend on the logarithm of the wettability. 
By the virtue of the model a combination of a spatially varying wettability as well as the addition of thermal fluctuation is studied using a dewetting thin film.
Depending on the chosen function of the wettability it is possible to create numerical experiments where the deterministic simulation is almost indistinguishable from the fluctuating one.

In Chap.~\ref{chapter:third_paper} a simple toy model is developed to study dewetting dynamics under the influence of ``switchable'' substrates.
The model is build on the idea that the wettability, e.g. the equilibrium contact angle $\theta_{\text{eq.}}$, can be changed based on external stimuli. We assume that the these stimuli can be modelled by an arbitrary function.
In the here discussed case the arbitrary function is a trigonometric function ($\theta(x,t) \propto \sin(x t)$) that admits not only a spatial gradient but a time dependent behaviour.
Due to the wettability gradient or capillarity fluid is driven to regions of high wettability, or low contact angles.
Therefore the stationary state for vanishing time dependency is simply an array of droplets formed in the minima of the contact angle field.
Adding a temporal component to the contact angle field introduces a stabilizing effect to the dewetting film, which lead to an increase in rupture time.
Most interestingly is the morphological transition that emerges when the dynamics of the contact angle field becomes fast as compared to the capillary retraction.
This means that the film finds itself in a metastable energy minimum where all fluid accumulates in rivulets.
Metastable because the rivulets are prone to an instability. 
In fact, independent of the parameters we vary rupture of these rivulets can be observed. 
The stationary state after the breakup is again a multi-droplet state, however with less droplets than in the static case.

Most surprisingly to the author of this thesis the lattice Boltzmann model for thin film hydrodynamics does in fact work very well. 
It is capable of performing numerical experiments in the desired fluid dynamic regimes ($Re < 1$, $Ca < 1$) without superficial parameter tuning. 
Although the translation of results into SI units may not be that insightful, because the method build on the Boltzmann equation where length and time scales are rather small.  
However matching with characteristic quantities such as the most unstable mode of the film ($q_0$) does allow comparing with experiments and theory.
The here presented work should be understood as a baseline to what the model is currently able to do.
Due to the time limitation of a PhD several ideas and features such as the dynamics of surfactants, soft substrates, non-Newtonian liquids, multiple liquids or the inclusion of (active) collides have to be postponed to the outlook.  
Is it possible to address these interesting topics with the here presented model?
The clear answer to this question is, \textit{yes}. 
It is possible to address these research areas with \textit{Swalbe.jl} and the thin film lattice Boltzmann model.
Among this short list of topics first steps were taken to study the dynamics of thin liquid films with immersed active colloids.
While the model does not allow to have a height resolved distribution of colloids inside the film, it in fact allows having a scalar density of colloids.
Depending on the activity of these colloids the density can be coupled to the film with e.g. a modified film pressure.
The dynamics of the colloids on the other hand allow for their own system of equations.
Within the limit of slow change of colloidal density (flowing with the film) the integration of evolution equation can simply be coupled to the lattice Boltzmann time iteration. 

While this is just a single example of a new and exciting problem, the open and freely available nature of the solver should allow for much of these projects.
Similar to the development of complex software, scientific problems are always in an evolving state.
One theory is not wrong but rather replaced by another one that creates a deeper understanding.
Therefore although this is the scientific end of my work as a PhD student, I personally hope that possible successors will achieve novel and interesting results based on the here presented model and ideas.\\
\\

\textit{``Talent is cheap, you have to be possessed or obsessed, rather. You really have to feel like you cannot not do art, and that is something you can’t will.''} - John Baldessari.

