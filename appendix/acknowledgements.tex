\chapter{Acknowledgments}

``\textit{Doing a PhD is not a sprint but a marathon}``, is an often written quote.
I did not hear that one personally, but I do understand it.
Coming with a master’s degree and having an idealistic view of science and academia is motivating and encouraging.
But this initial emotional rush quite naturally fades out with time.
To keep the train on track and finally write this thesis I have to thank many people that helped me along the way.

I'd like to start by thanking Dr. Andrea Scagliarini. 
He was the initiator of the project that led to my thesis. 
Even though he was only physically around for the first month of my PhD he kept helping me out for the full duration.
Needless to say that not even Corona and the therefore resulting home office period did not stop our collaboration.
His deep understanding of hydrodynamics and physics in general was always very insightful.  
I am especially thankful for his positive review of my HPC-europa proposal.
Due to this I had a fruitful and productive research stay in Rome.

Of course, I could only reach this point due to constant support of my supervisor Prof. Dr. Jens Harting.
He was brave enough to hire someone that had little to no prior knowledge with fluid dynamics and scientific computing.
While it was not always easy, because different characters see different problems I thank him wholeheartedly for his supervision.
He offered me all the freedom to learn new things and being creative with computational experiments.
His trust allowed me to visit and present my work at several international conferences and let me engage with the community.

As my mood was not always the best during this time I have to thank the one person who always stood with me, Leonie.
Not only had she to listen to my rants about work, but she also endured my negative attitude during setbacks.
She was always supportive during this time and kept me functional.
Due to her I got a very different perspective on problem solving and acquired a little taste for art.

I also thank my whole family who supported me not only for the last few years but during my whole life.
For the longest time in my life I could ask my parents for help and without hesitation they would have helped me.
That changed with my relocation to Nürnberg as there was \textit{suddenly} distance between us.
They tried to keep up with me and my life, with more and less success.
On the same note I thank both my grandparents for being in my life.
Both care deeply for me and I am very lucky that they are still around.
Supporting me with care packages I received from time to time, which were always filled with delicious goods.
During the time of writing this work, my grandfather Peter Tschuchnik has passed away after a short intensive sickness.
I have so many beautiful memories being in Zeltweg, thank you for that Opa I miss you.  

Thanks to my friend, Clemens, and my brother Thomas who although in a different country I am in contact on an almost daily base.
The same holds for my colleagues and friends that I made during my time in Nürnberg.
To name just a few Nico, David, Daniel, Nadiia, Alex, Mathias, Marcello, Björn, Olivier, Lei, Thomas and Othmane you made the office always a nice place to be.
One person that deserves his own sentence is Manuel our former IT admin and live saver.
Thanks to his engaging lectures on coding and scientific computing as well as version control I quickly caught up with building my own usable software. 

There were so many people who supported me on my journey and I am very sorry for those who I have not mentioned here.
I could not have done this alone and I know that. 
Therefore I am in dept to so many people.
Thank you very much Leonie, my family, friends, Andrea, Jens and many more.